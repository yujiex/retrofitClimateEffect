% note: XX means needs modification
\documentclass[12pt]{article}
\linespread{1.3}
\usepackage{scrextend}
\usepackage{hyperref}
% \usepackage{enumitem}
\usepackage{enumerate}
\usepackage{changepage,lipsum,titlesec, longtable}
\usepackage{cite}
\usepackage{comment, xcolor}
\usepackage{csvsimple}
\usepackage[pdftex]{graphicx}
  \graphicspath{{images/}, {images/stat/}}
  \DeclareGraphicsExtensions{.pdf,.jpeg,.png, .jpg}
\usepackage[cmex10]{amsmath}
\usepackage{array} 
\usepackage{subfigure} 
\usepackage{placeins} 
\usepackage{amsfonts}
\usepackage{pifont}% http://ctan.org/pkg/pifont
\usepackage{fancyvrb}
\usepackage{lipsum}
\usepackage{booktabs}
\usepackage{multirow}
\usepackage{pdflscape}
% \usepackage{minted}
% \usepackage{natbib}
% \usepackage{apacite} 

\newcommand{\cmark}{\ding{51}}%
\newcommand{\xmark}{\ding{55}}%
\newcommand{\grey}[1]{\textcolor{black!30}{#1}}
\newcommand{\red}[1]{\textcolor{red!50}{#1}}
\newcommand{\fref}[1]{Figure~\ref{#1}}
\newcommand{\tref}[1]{Table~\ref{#1}}
\newcommand{\eref}[1]{Equation~\ref{#1}}
\newcommand{\sref}[1]{Section~\ref{#1}}
\newcommand\myworries[1]{\textcolor{red}{\{#1\}}}
\newenvironment{loggentry}[2]% date, heading
{\noindent\textbf{#2}\marginnote{#1}\\}{\vspace{0.5cm}}

\oddsidemargin0cm
\topmargin-2cm %I recommend adding these three lines to increase the
\textwidth16.5cm %amount of usable space on the page (and save trees)
\textheight23.5cm

\makeatletter
\renewcommand\paragraph{\@startsection{paragraph}{4}{\z@}%
            {-2.5ex\@plus -1ex \@minus -.25ex}%
            {1.25ex \@plus .25ex}%
            {\normalfont\normalsize\bfseries}}
\makeatother
\setcounter{secnumdepth}{4} % how many sectioning levels to assign numbers to
\setcounter{tocdepth}{4}    % how many sectioning levels to show in ToC

\begin{document}
\title{Data processing}
\maketitle
\tableofcontents
\newpage
\section{EUAS data cleaning}

Fields
\begin{itemize}
\item BLDGCAT: building categories. According to
  \url{https://www.epa.gov/sites/production/files/2017-12/documents/tom_burke_gsa_fgcwebinar_12_13_17_002.pdf},
  \begin{itemize}
  \item government owned (A, B)
  \item non-goal leased (C, D)
  \item Goal Energy Intensive (I)
  \item Reimbursable (E)
  \item Optional Designations ( EISA Cov Fac, Metered, LPOE, etc)
  \end{itemize}
\end{itemize}

We'll assume buildings in category C or D are leased, others are owned
\section{Title}

\section{Retrofits}
The retrofit records are from three sources: xx and xx

EUAS energy data has only month and year, but not a specific duration, we will
assume the center of the billing period being the 15th. This center date will be
used to filter the appropriate pre and post retrofit period.

For the retrofitted buildings, we consider the two years prior to the
documented ``Substantial Completion Date'' as the retrofit implementation
period, and remove them from the analysis. The three-year before the
implementation period, and the three-year after the ``Substantial Completion
Date'' are used to compute the outcome variable, the change of average monthly
consumption (or consumption per sqft).

For the non-retrofitted buildings, a fake retrofit date is chosen from a random
sample of the actual retrofit dates in the retrofitted buildings, and the pre
and post period are computed similarly as the retrofitted buildings.

\newpage
\bibliographystyle{plain}
% \bibliography{my2Citation}
\end{document}