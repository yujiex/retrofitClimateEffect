% note: XX means needs modification
\documentclass[12pt]{article}
\linespread{1.3}
\usepackage{scrextend}
\usepackage{hyperref}
% \usepackage{enumitem}
\usepackage{enumerate}
\usepackage{changepage,lipsum,titlesec, longtable}
\usepackage{cite}
\usepackage{comment, xcolor}
\usepackage{csvsimple}
\usepackage[pdftex]{graphicx}
  \graphicspath{{images/}, {images/stat/}}
  \DeclareGraphicsExtensions{.pdf,.jpeg,.png, .jpg}
\usepackage[cmex10]{amsmath}
\usepackage{array} 
\usepackage{subfigure} 
\usepackage{placeins} 
\usepackage{amsfonts}
\usepackage{pifont}% http://ctan.org/pkg/pifont
\usepackage{fancyvrb}
\usepackage{lipsum}
\usepackage{booktabs}
\usepackage{multirow}
\usepackage{pdflscape}
\usepackage{float}
% \usepackage{minted}
% \usepackage{natbib}
% \usepackage{apacite} 

\newcommand{\cmark}{\ding{51}}%
\newcommand{\xmark}{\ding{55}}%
\newcommand{\grey}[1]{\textcolor{black!30}{#1}}
\newcommand{\red}[1]{\textcolor{red!50}{#1}}
\newcommand{\fref}[1]{Figure~\ref{#1}}
\newcommand{\tref}[1]{Table~\ref{#1}}
\newcommand{\eref}[1]{Equation~\ref{#1}}
\newcommand{\sref}[1]{Section~\ref{#1}}
\newcommand\myworries[1]{\textcolor{red}{\{#1\}}}
\newenvironment{loggentry}[2]% date, heading
{\noindent\textbf{#2}\marginnote{#1}\\}{\vspace{0.5cm}}

\oddsidemargin0cm
\topmargin-2cm %I recommend adding these three lines to increase the
\textwidth16.5cm %amount of usable space on the page (and save trees)
\textheight23.5cm

\makeatletter
\renewcommand\paragraph{\@startsection{paragraph}{4}{\z@}%
            {-2.5ex\@plus -1ex \@minus -.25ex}%
            {1.25ex \@plus .25ex}%
            {\normalfont\normalsize\bfseries}}
\makeatother
\setcounter{secnumdepth}{4} % how many sectioning levels to assign numbers to
\setcounter{tocdepth}{4}    % how many sectioning levels to show in ToC

\begin{document}
\title{Data processing}
\maketitle
\tableofcontents
\newpage
\section{EUAS data cleaning}

Fields
\begin{itemize}
\item BLDGCAT: building categories. According to
  \url{https://www.epa.gov/sites/production/files/2017-12/documents/tom_burke_gsa_fgcwebinar_12_13_17_002.pdf},
  \begin{itemize}
  \item government owned (A, B)
  \item non-goal leased (C, D)
  \item Goal Energy Intensive (I)
  \item Reimbursable (E)
  \item Optional Designations ( EISA Cov Fac, Metered, LPOE, etc)
  \end{itemize}
\end{itemize}

We'll assume buildings in category C or D are leased, others are owned

\subsection{Building Type}
Building types affects when and how buildings are operated, thus affecting its
energy consumption pattern. It also might impact some other physical aspects of
the building, and its retrofit decisions. In the GSA data set, there are 4
different sources of building types, one from Energy Star Portfolio Manager, one
from a excel file sent to us by the GSA team, and the 2012 and 2014 EPA National
Building Competition records from online sources.

Among these sources, many have disagreement on what type a building belongs to.
We first removed the type called ``Other'' and ``All other'', as these labels
provides no type information. Secondly, we have noticed that ``Office'' seems to
be used as a default type, and the name of some buildings suggest they are of a
different category. For example, we've seen parking lots, and heating plants
labeled as offices. As a result, when we see a building with multiple type
labels, we will give higher priority to other types labels. For example, when a
building is labeled warehouse and office, we'll consider it to be warehouse.

Among the building types, the ``Utility Systems'' are those acting as power
plants. The energy consumption of these buildings are dependent on their
downstream consumers, thus will be removed from the analysis. To identify
mis-labeled power plants, we first collected a set of key words suggesting a
building is likely to be a power plant from the set of buildings already labeled
``Utility Systems'', and then reclassified other buildings with these key words
as ``Utility Systems'' as well.

\subsection{Retrofits}
The retrofit records are from three sources: xx and xx

EUAS energy data has only month and year, but not a specific duration, we will
assume the center of the billing period being the 15th. This center date will be
used to filter the appropriate pre and post retrofit period.

For the retrofitted buildings, we consider the two years prior to the
documented ``Substantial Completion Date'' as the retrofit implementation
period, and remove them from the analysis. The three-year before the
implementation period, and the three-year after the ``Substantial Completion
Date'' are used to compute the outcome variable, the change of average monthly
consumption (or consumption per sqft).

For the non-retrofitted buildings, a fake retrofit date is chosen from a random
sample of the actual retrofit dates in the retrofitted buildings, and the pre
and post period are computed similarly as the retrofitted buildings.

We filtered buildings with at least 12 monthly energy data, and with an average
of non-zero consumption. Then we restricted the study to buildings whose types
appear in both the retrofitted and un-retrofitted group for at least 10 times.
This leaves us with the office and courthouses.

For each period, we use the majority of the building category as the category
for this period. The ties are broken randomly. After checking, this won't affect
the derived ownership (no ties appear between category C/D or other categories).

Building type is from the file called ``Entire GSA Building Portfolio.xls''.
After a matching step, there are 128 buildings with known built year, 489 with
unknown century (as the year is recorded with two digits, so for some records,
it's possible to be 19xx or 20xx), and 846 unknown. Gave up on getting the built
year information for now.

May be able to get the list of historic buildings from \url{https://www.gsa.gov/node/95?actionParameter=displayAllBuildings&buildingVO.searchBy=CITY&state.id=0&__multiselect_selectedCities=&__multiselect_selectedBuildings=&buildingNos=AK0001ZZ&buildingNos=AK0005AK&buildingNos=AK0013ZZ&buildingNos=AK0022ZZ&__multiselect_buildingNos=&__multiselect_selectedArchiects=&buildingVO.buildingType.id=-1&buildingVO.yearBegin=&buildingVO.yearFinish=}.

LEED award date and level are retrieved from points are retrieved from
\url{http://www.gbig.org/collections/14796/buildings?page=1} for quality control
(even if there are other LEED data files given to us previously).

We match LEED project by address. Some addresses are matched to multiple
buildings, we resolve this by manually look up building names and see if it
matched the project description. The building name lookup file is from \url{https://raw.githubusercontent.com/Sreeramk/gsa-hackathon-t4/master/Data/costs.csv}

\section{Weather}
The daily min and max temperature, and daily precipitation are downloaded from NOAA using R package rnoaa.
\subsection{Daily temperature}

HI0011ZZ cannot find nearby stations within 20km radius.

when restricting a 36 months pre-retrofit energy, we are left with 280 retrofitted
and 283 un-retrofitted buildings. If not filter, we have 286 retrofitted and 291 non-retrofitted.

The bin representation of 10 degree bins are used.

\subsection{Daily precipitation}
The bin temperature with 5mm width are used.

\section{Climate}
We will produce an estimate of each individual model, and a weighted average of
all model estimates using the weights from Sanderson and Wehner (cite) table B.2.

The climate data of daily min and daily max temperature (tmin and tmax) for the
duration of 2090-2099 is downloaded from the bias corrected downscaled climate
scenario daily data (BCCA) of various climate models and runs from the CMIP5
multi-model ensemble archive. The data has a spatial resolution of 1/8 degrees.

The NEX-GDDP dataset has a resolution of 0.25 degrees, so I used the bcca data
set with finer spatial resolution.

The continental US region climate data are downloaded from
\url{https://gdo-dcp.ucllnl.org/downscaled_cmip_projections/dcpInterface.html#Projections:%20Subset%20Request}.
Following is a visualization of the grid points of one time snapshot of one
model. 20 models of mid and end century data are downloaded.

\begin{figure}[H]
  \centering
  \includegraphics[width=0.6\textwidth]{../images/climate_scenario_grid_us.png}
  \caption[daily max temperature grid]{Daily max temperature of one model one
    day for the whole US}
  \label{fig:climate_scenario_grid_us}
\end{figure}

The following is a zoomed-in plot of a sub-region.
\begin{figure}[H]
  \centering
  \includegraphics[width=0.6\textwidth]{../images/zoomed_climate_scenario_grid_us.png}
  \caption[daily max temperature grid]{Daily max temperature of one model one
    day for Pennsylvania}
  \label{fig:zoomed_climate_scenario_grid_us}
\end{figure}

We downloaded 19 models for RCP45 and RCP85 for three different period:
2011-2014, 2050-2099, and 2090-2099. The models are present in all scenarios and period.

The following 13 buildings have
no data as a result of failure in interpolation: TX1979ZZ, WA7546ZZ, FL0019ZZ,
FL2917ZZ, FL3150ZZ, MA0134ZZ, MA5311ZZ, MA5780ZZ, MI1657ZZ, OH0192ZZ, TX0079ZZ,
TX0080ZZ, TX0081ZZ

Among the retrofit data set, three buildings don't have the cmip5 climate
scenario data, as they are at the edge or outside of the continental US, thus
the interpolation failed.

\begin{figure}[H]
  \centering
  \includegraphics[width=0.8\textwidth]{../images/retrofit_cmip5_loc.png}
  \caption[cmip5 availability for retrofit data set]{Climate scenario data
    availability for the retrofit data set}
  \label{fig:retrofit_cmip5_loc}
\end{figure}


Note that some buildings share the same latitude and longitude location.
\section{Plots}
The post-retrofit average monthly consumption seems lower than that of the
pre-retrofit period.
\begin{figure}[H]
  \centering
  \includegraphics[width=0.5\textwidth]{../images/diff_mean_energy.png}
  \caption{Difference in average monthly consumption}
\end{figure}

Comparing the retrofitted and the non-retrofitted, the pre and post consumption.

\newpage
\bibliographystyle{plain}
% \bibliography{my2Citation}
\end{document}