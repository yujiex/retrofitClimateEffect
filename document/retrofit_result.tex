% note: XX means needs modification
\documentclass[12pt]{article}
\linespread{1.3}
\usepackage{scrextend}
\usepackage{hyperref}
% \usepackage{enumitem}
\usepackage{enumerate}
\usepackage{changepage,titlesec}
\usepackage{cite}
\usepackage{comment, xcolor}
\usepackage{csvsimple}
\usepackage[pdftex]{graphicx}
  \graphicspath{{images/}, {images/stat/}}
  \DeclareGraphicsExtensions{.pdf,.jpeg,.png, .jpg}
\usepackage[cmex10]{amsmath}
\usepackage{array} 
% \usepackage{subfigure} 
\usepackage{subcaption} 
% \usepackage{placeins} 
\usepackage{amsfonts}
\usepackage{pifont}% http://ctan.org/pkg/pifont
\usepackage{fancyvrb}
% \usepackage{lipsum}
\usepackage{booktabs}
% \usepackage{multirow}
% \usepackage{pdflscape}
\usepackage{float}
% \usepackage{minted}
% \usepackage{natbib}
% \usepackage{apacite} 

\newcommand{\cmark}{\ding{51}}%
\newcommand{\xmark}{\ding{55}}%
\newcommand{\grey}[1]{\textcolor{black!30}{#1}}
\newcommand{\red}[1]{\textcolor{red!50}{#1}}
\newcommand{\fref}[1]{Figure~\ref{#1}}
\newcommand{\tref}[1]{Table~\ref{#1}}
\newcommand{\eref}[1]{Equation~\ref{#1}}
\newcommand{\sref}[1]{Section~\ref{#1}}
\newcommand\myworries[1]{\textcolor{red}{\{#1\}}}
\newenvironment{loggentry}[2]% date, heading
{\noindent\textbf{#2}\marginnote{#1}\\}{\vspace{0.5cm}}

\oddsidemargin0cm
\topmargin-2cm %I recommend adding these three lines to increase the
\textwidth16.5cm %amount of usable space on the page (and save trees)
\textheight23.5cm

\makeatletter
\renewcommand\paragraph{\@startsection{paragraph}{4}{\z@}%
            {-2.5ex\@plus -1ex \@minus -.25ex}%
            {1.25ex \@plus .25ex}%
            {\normalfont\normalsize\bfseries}}
\makeatother
\setcounter{secnumdepth}{4} % how many sectioning levels to assign numbers to
\setcounter{tocdepth}{4}    % how many sectioning levels to show in ToC

\begin{document}
\title{Data processing}
\maketitle
\tableofcontents
\newpage
\section{Introduction}
\section{Background}
\section{Data source}
\subsection{Energy Data} The monthly energy consumption data of the GSA
portfolio is from the Energy Usage Analysis System (EUAS)~\cite{euas2019}. It
contains monthly consumption data of a variety of fuel types including:
electricity, natural gas, steam, oil, coal, and chilled water. 
                                    
The EUAS data set contains energy data from fiscal year 1973 to 2019. As the
data is updated on 2019 March, we restrict the analysis to 2018 and before.

\fref{fig:euas_web_num_building} shows the number of buildings in each Fiscal
Year. We can see there's some discrepancy in the energy before 1990, thus we
restrict our analysis to data after 1989. \fref{fig:euas_web_num_year} shows the
distribution of the number of years of data for individual buildings. We can see
the majority of buildings has less than 10 years of data, but there's also a
substantial number of buildings with long-term data over 20 or 30 years.

\begin{figure}[H]
  \centering
  \includegraphics[width=0.7\textwidth]{../images/euas_web_num_building.png}
  \caption{Number of buildings in each fiscal year}
  \label{fig:euas_web_num_building}
\end{figure}

\begin{figure}[H]
  \centering
  \includegraphics[width=0.7\textwidth]{../images/euas_web_num_year.png}
  \caption{Distribution of how many years of data a building has}
  \label{fig:euas_web_num_year}
\end{figure}

\fref{fig:buildingloc_1990_2018} plots the geographical locations of buildings
in 1990 and in 2018. This together with \fref{fig:euas_web_num_building} shows
that the number of buildings in the GSA portfolio gradually decreases over the
years.

\begin{figure*}[t!]
    \centering
    \begin{subfigure}[t]{0.7\textwidth}
        \centering
        \includegraphics[width=1.0\textwidth]{../images/buildingloc_1990.png}
        \caption{Buildings in 1990}
    \end{subfigure}
    \begin{subfigure}[t]{0.7\textwidth}
        \centering
        \includegraphics[width=1.0\textwidth]{../images/buildingloc_2018.png}
        \caption{Buildings in 2018}
    \end{subfigure}
    \caption{Buildings in 1990 and 2018}
    \label{fig:buildingloc_1990_2018}
\end{figure*}

\subsection{Weather Data}
The daily min and max temperature, and daily precipitation are downloaded from
NOAA using R package rnoaa~\cite{rnoaa2017}. Each building is matched to 5
nearby weather stations with at least 80\% data each given year. The weather at
each building location will be an inverse distance weighted average of the
nearby stations. On average, the stations are 4 to 20 miles from the building,
and at least one station is within 20 miles from a building at all time.

\subsection{Climate Normals}
Annual cooling and heating degree days 1981-2010 climate normal NOAA
FTP~\cite{NOAAclimateNormal2020}. For each building, we find all weather
stations within 20 mile radius to the building, and use the inverse-distance
weighted average of the climate normals of all the weather stations as its
climate normal.

\subsection{Climate Change Scenarios}
The climate data of daily min and daily max temperature (tmin and tmax), and
daily precipitation of the bias corrected downscaled climate scenario data
(BCCA) for the continental US during 2011-2014, 2050-2059, and 2090-2099 are
downloaded from the of 19 climate models from the ``Downscaled CMIP3 and CMIP5
Climate and Hydrology Projections'' archive~\cite{Reclamation2013}. The data has
a spatial resolution of 1/8 degrees. The NEX-GDDP dataset has a resolution of
1/4 degrees, coarser than the former, thus I used the former data set. There are
overall 21 climate models, but as we want to have models appear in all three
period, we restrict the analysis for the 19 models.

% fixme: need to list the models

Following is a visualization of the grid points of one time snapshot of one
model.

\begin{figure}[H]
  \centering
  \includegraphics[width=0.6\textwidth]{../images/climate_scenario_grid_us.png}
  \caption[daily max temperature grid]{Daily max temperature of one model one
    day for the whole US}
  \label{fig:climate_scenario_grid_us}
\end{figure}

The following is a zoomed-in plot of a sub-region.
\begin{figure}[H]
  \centering
  \includegraphics[width=0.6\textwidth]{../images/zoomed_climate_scenario_grid_us.png}
  \caption[daily max temperature grid]{Daily max temperature of one model one
    day for Pennsylvania}
  \label{fig:zoomed_climate_scenario_grid_us}
\end{figure}

% \begin{figure}[H]
%   \centering
%   \includegraphics[width=0.8\textwidth]{../images/retrofit_cmip5_loc.png}
%   \caption[cmip5 availability for retrofit data set]{Climate scenario data
%     availability for the retrofit data set}
%   \label{fig:retrofit_cmip5_loc}
% \end{figure}

\subsection{Building characteristics}
Building size (gross square foot), category, region number are retrieved from
the EUAS data set. Building ownership is derived from building category.

Building type is acquired from multiple data sources, including the
output excel table from Energy Star Portfolio Manager~\cite{portfolioManager2016}, 
a excel data file from the GSA team, and the 2012 and 2014 EPA National
Building Competition records~\cite{nationalCompetitionEPA2012, nationalCompetitionEPA2014}.

Indicator for historic building is from the GSA Historic Building Database~\cite{GSAhistoricBuilding2020}.

\subsection{Retrofits}
The retrofit records are from four excel files acquired from the GSA team:
``Light-Touch\_M\&V'', ``Portfolio HPGB Dashboard'' and
``GSAlink\_Buildings\_First\_55'', and ``ScopePortfolioReport\_20160105-7''.

\section{Summary statistics}

% In the retrofit analysis, we control for the following covariates
% \begin{itemize}
% \item Pre-retrofit actions: an indicator of whether certain type of retrofit
%   action $A$ happen during or before the target retrofit $A'$
% \item Pre-retrofit average monthly consumption ($kbtu/sqft$): an indicator of whether certain type of retrofit
%   action $A$ happen during or before the target retrofit $A'$
% \end{itemize}

We plotted the energy consumption trend for the retrofit impact analysis data
set. We can see average kBtu/sqft consumption for the retrofitted building is
slightly lower than the non-retrofitted buildings prior to retrofits. However
the difference between them widens after the retrofit, especially during the
summer months.

\begin{figure}[H]
  \centering
  \includegraphics[width=1.0\textwidth]{../images/energy_retrofit_status_highprop.png}
  \caption[Energy trend retrofit study]{Energy consumption trend for retrofit
    building study set}
  \label{fig:energy_retrofit_status}
\end{figure}

The following table shows the retrofit building count
\input{../tables/detail_retrofit_building_count.tex}

The following checks the balance between the retrofitted and non-retrofitted buildings.

\input{../tables/numeric_feature_balance_compare.tex}
\input{../tables/indicator_feature_balance_compare.tex}

We estimated propensity score with logistic regression. After removing
records with propensity score that are below 0.05 or above 0.95, the covariate
balance is as follows

\input{../tables/numeric_feature_balance_compare_highprop.tex}
\input{../tables/indicator_feature_balance_compare_highprop.tex}

The distribution of the propensity score is shown in 

\begin{figure}[H]
  \centering
  \includegraphics[width=0.7\textwidth]{../images/propensity_by_treat_highprop.png}
  \caption[Propensity score distribution]{Propensity score difference}
  \label{fig:propensity}
\end{figure}

We examined the difference of energy consumption, and energy consumption per
square foot difference of four different fuel types.

\begin{figure}[H]
  \centering
  \includegraphics[width=0.7\textwidth]{../images/diff_mean_energy.png}
  \caption[Energy trend retrofit study]{Pre and post retrofit difference in average monthly
    consumption (kBtu)}
  \label{fig:diff_mean_energy}
\end{figure}
\begin{figure}[H]
  \centering
  \includegraphics[width=0.7\textwidth]{../images/diff_mean_eui.png}
  \caption[Energy trend retrofit study]{Pre and post retrofit difference in average monthly
    consumption per square foot (kBtu/sqft)}
  \label{fig:diff_mean_eui}
\end{figure}

The retrofitted buildings is slightly larger.

\begin{figure}[H]
  \centering
  \includegraphics[width=0.4\textwidth]{../images/retrofit_building_size.png}
  \caption[Building size]{Building size distribution}
  \label{fig:retrofit_building_size}
\end{figure}
\begin{figure}[H]
  \centering
  \includegraphics[width=0.6\textwidth]{../images/retrofit_building_type.png}
  \caption[Building type]{Building type}
  \label{fig:retrofit_building_type}
\end{figure}
\begin{figure}[H]
  \centering
  \includegraphics[width=0.6\textwidth]{../images/retrofit_building_historic.png}
  \caption[Historic buildings]{Number of historic and non-historic buildings}
  \label{fig:retrofit_building_historic}
\end{figure}
\begin{figure}[H]
  \centering
  \includegraphics[width=0.8\textwidth]{../images/retrofit_pre_consumption.png}
  \caption[Pre-retrofit energy]{Pre-retrofit average monthly consumption}
  \label{fig:retrofit_pre_consumption}
\end{figure}
\begin{figure}[H]
  \centering
  \includegraphics[width=0.4\textwidth]{../images/retrofit_pre_climate_normal.png}
  \caption[Pre-retrofit climate normal]{Pre-retrofit climate normal}
  \label{fig:retrofit_pre_climate_normal}
\end{figure}
\begin{figure}[H]
  \centering
  \includegraphics[width=1.0\textwidth]{../images/retrofit_pre_temperature_bin.png}
  \caption[Pre-retrofit weather]{Pre-retrofit temperature distribution}
  \label{fig:retrofit_pre_temperature_bin}
\end{figure}
\begin{figure}[H]
  \centering
  \includegraphics[width=1.0\textwidth]{../images/retrofit_pre_action.png}
  \caption[Pre-retrofit weather]{Other retrofits co-exist or prior to the target
  retrofit}
  \label{fig:retrofit_pre_action}
\end{figure}
\begin{figure}[H]
  \centering
  \includegraphics[width=0.9\textwidth]{../images/retrofit_climate_scenario.png}
  \caption[Climate scenario]{Temperature distribution for different climate
    scenarios and models}
  \label{fig:retrofit_climate_scenario}
\end{figure}

\section{Models and results}
\subsection{Causal forest}
\begin{figure}[H]
  \centering
  \includegraphics[width=0.9\textwidth]{../images/retrofit_effect_cf.png}
  \caption[Causal forest prediction]{Distribution of treatment effect}
  \label{fig:retrofit_effect_cf}
\end{figure}

\begin{figure}[H]
  \centering
  \includegraphics[width=0.9\textwidth]{../images/var_importance_cf_envelope_gas.png}
  \caption[Causal forest variable importance]{Covariate variable importance}
  \label{fig:var_importance_cf_envelope_gas}
\end{figure}

\subsection{Synthetic control}
\section{Discussion / Interpretation}
\section{Conclusion}
\section{Acknowledgment}
We acknowledge the World Climate Research Programme's Working Group on Coupled
Modelling, which is responsible for CMIP, and we thank the climate modeling
groups (listed in \tref{tab:climate_models} of this paper) for producing and making available
their model output. For CMIP the U.S. Department of Energy's Program for Climate
Model Diagnosis and Intercomparison provides coordinating support and led
development of software infrastructure in partnership with the Global
Organization for Earth System Science Portals.

\section{Appendix I: Data cleaning}
\subsection{Monthly energy data from EUAS} The consumption data is in the table
``mar.energyutilization.xlsx'', downloaded from ~\cite{euas2019}. The data file
has the following fields: BLDGNUM, FYR, FMONTH, KWHRAMT, KWDMDAMT, STEAMAMT,
GASAMT, OILAMT, COALAMT, KWHRCOST, KWDMDCOST, STEAMCOST, GASCOST, OILCOST,
COALCOST, KWHRASRC , KWHRDMDASRC, STEAMASRC, GASASRC, OILASRC, COALASRC,
KWHRCSRC, KWDMDCSRC, STEAMCSRC, GASCSRC, OILCSRC, COALCSRC, CHILLWTRAMT,
CHILLWTRCOST, CHILLWTRCSRC, CHILLWTRASRC, REMARKS, WTRAMT, WTRCOST, WTRCSRC,
WTRASRC , REELECAMT, REELECCOST, REELECCSRC, REELECASRC, REGASAMT, REGASCOST,
REGASCSRC, REGASASRC, RECWTRAMT, RECWTRCOST, RECWTRCSRC, RECWTRASRC, REOILAMT,
REOILCOST, REOILCSRC, REOILASRC, WATER\_REMARKS. Some column names are self
explanatory, for example, the columns ending with ``AMT'' are likely to be
energy consumptions, and the ones ending with ``COST'' are likely utility
expenses in dollars. Other columns have non-trivial names, for example, the ones
ending with ``CSRC'' and ``ASRC''. These columns are excluded from the analysis.
The columns with prefix ``RE'' are likely to be renewable energy. However we are
uncertain whether they are consumption or output, thus are also excluded from
the analysis.

The following table shows the cleaning steps and the remaining records of the
monthly energy.

\begin{table}[H]
  \caption{Cleaning steps and records remaining}
  \centering
  \fontsize{10}{12}\selectfont
  \begin{tabular}{l|p{6cm}|l|l}
  \toprule
    step & operation &number of buildings& number of records\\
  \midrule
    1& Initial& 3494 &6166413\\
    2& Remove records with negative consumption & 3494 & 6166411\\
    3& Remove the records for each fuel and Fiscal Year where total consumption is zero & 3274 & 1181741\\
    4& The study does not analyze water consumption, so removed them & 3271 & 1062104\\
    5& Has state city location \footnote{We have seen cases where the initials of the building ID do not correspond to the state abbreviation of the state the building locates in, so we remove records which we cannot find a definite information about its state} & 3253 & 1060939\\
    6& With building size, category, and region information & 3253 & 1060939\\
    7& Gross square foot is larger than 1 sqft & 3182 & 1052392\\
    8& With valid zip code or city information so that the latitude longitude could be retrieved & 3167 & 1050388\\
    9& Within continental US (not in AK, HI, VI, GU, PR, VI, MP) & 3058 & 1027178\\
    10& With energy from 1990 to 2018 & 499 & 173652\\
  \bottomrule
  \end{tabular}
  \label{tab:cleaning_steps}
\end{table}

The consumption data are converted to kBtu using the table
``mar.unitconversions.xlsx'' from~\cite{euas2019}.

\subsection{Weather Data}

\subsection{Climate Data}

\subsection{Building location} The building state, city, zipcode information are in
the ``mar.facilitybuildings.xlsx'' file of the EUAS data set~\cite{euas2019}.
3181 buildings appear in this data file. We first matched the zipcode and state
to a zipcode centroid location lookup table,
``us-zip-code-latitude-and-longitude.csv'' from \cite{civicSpace2020} and the
\texttt{us\_zipcodes()} output from the R package USAboundaries \cite{USAboundaries2018},
to get the latitude longitude location of the zipcode. The building is assumed
to located at the centroid of its zip code region. We need the latitude
longitude of a building in order to retrieve the local weather data, thus we
think the zip code resolution is enough. 3055 buildings are matched up in this
process. We suspect these non-matched buildings have some errors in the
documented zip code information. We looked up the centroid of cities the
non-matched building located in, and found their latitude longitude with the
\texttt{us\_cities()} function from the R package USAboundaries \cite{USAboundaries2018},
and some manual lookup with Google Maps. Finally 3167 out of the 3181 records
are matched to a latitude longitude location, with 3055 matched to its zip code,
and 112 buildings matched to its city centroid.

\subsection{Building gross square foot, and ownership} Acquired from the
``mar.fyrbuilding.xlsx'' file in the EUAS data set~\cite{euas2019}. This file
has building id (BLDGNUM), the region a building belongs to (REGNNUM), fiscal
year (FYR), building category (BLDGCAT), gross square foot (GROSSSQFT), whether
a building is goal tracked (GOAL TRACKED). This file is joined to the monthly
consumption file, by building id and fiscal year. When the building category or
gross square foot are missing for certain fiscal year, it is filled with the
most recent year where data is available.

The ownership is derived from the building category (BLDGCAT). According to
\cite{Burke2017}, the building category definitions are. According to the
category designations, we'll assume buildings in category C or D are leased,
others are owned.

\begin{table}[H]
  \caption{Building category definition}
  \centering
  \fontsize{10}{12}\selectfont
  \begin{tabular}{l|l}
  \toprule
    BLDGCAT & Definition \\
    \midrule 
    A, B & government owned \\
    C, D & non-goal leased \\
    I & Goal Energy Intensive \\
    E & Reimbursable \\
    \bottomrule
  \end{tabular}
  \label{tab:building_cat}
\end{table}

\subsection{Building Type} Building types affects when and how buildings are
operated, thus affecting its energy consumption pattern. It also might impact
some other physical aspects of the building, and its retrofit decisions. In the
GSA data set, there are 4 different sources of building types, one from the
output from Energy Star Portfolio Manager~\cite{portfolioManager2016}, one from
a excel file from the GSA team, and the 2012 and 2014 EPA National
Building Competition records from online
sources~\cite{nationalCompetitionEPA2012, nationalCompetitionEPA2014}.

Among these sources, many have disagreement on what type a building belongs to.
We first removed the type called ``Other'' and ``All other'', as these labels
provides no type information. Secondly, we have noticed that ``Office'' seems to
be used as a default type, and the name of some buildings suggest they are of
some different category. For example, we've seen parking lots, and heating
plants labeled as offices. As a result, when we see a building with multiple
type labels, we will give higher priority to other types labels. For example,
when a building is labeled warehouse and office, we'll consider it to be
warehouse. A second metric for deciding among conflicting building type labels
is to choose the more specific type. For example, when a building is labeled as
both ``Public Service'' and ``Service'', we choose the ``Public services''
label. Also when a building is labeled as both ``Entertainment/PA'' and
``Museum'', we label them as ``Museum''.

The labeling of certain building types have slightly different wording among
different sources. We thus unified and relabeled certain building types (see
\tref{tab:building_type_relabel}). We conducted the following relabeling.
``Mailing Center/Post Office'' are all re-labeled as ``Office'', as they have
similar operation pattern and we can see they have relatively similar EUI
according to the table in~\cite{euiPortfolioManager2018}.

\begin{table}[H]
  \caption{Relabel of certain building types}
  \centering
  \fontsize{10}{12}\selectfont
  \begin{tabular}{l|l}
    \toprule
    Old&New\\
    \midrule
    Office Building & Office \\
    Mailing Center/Post Office & \\
    \midrule
    Non-Refrigerated Warehouse & Warehouse\\
    Storage Other than Bldg (OTB) & \\
    \midrule
    Other - Services & Service\\
    Other - Education & Education \\
    Other - Public Services & Public services \\
    CT/Office & Courthouse \\
    All Other & Other \\
    \bottomrule
  \end{tabular}
  \label{tab:building_type_relabel}
\end{table}

Among the building types, the ``Utility Systems'' buildings act as energy
suppliers to other buildings. The energy consumption of these buildings are
dependent on their downstream consumers, thus it is critical to identify these
power plant buildings and exclude them from the analysis. As is discussed above,
there are some mis-labelling issues in the building types, for example, the
DC0001ZZ building, is labeled as ``office'' in all our building type data
sources, while the building seems to be a heating plant in reality~\cite{dc0001zz2017}.
To identify mis-labeled power plants, we first collected a set of key words
suggesting a building is likely to be a power plant from the set of buildings
already labeled as ``Utility Systems'', and then reclassified other buildings with
these key words as ``Utility Systems'' as well. The keywords we selected are:
HEATING TUNNELS, DISTRI TUNNELS, STEAM SERV, POWER PLANT, BOILER PLANT,
MECHANICAL EQUIPMENT, GENERATOR, PUMPHOUSE, WATER TOWER, HOTA STM, HTG PLT,
HEATING PLANT, UTLY PLANT, BOILERHOUSE, ELEC SUB STA, PUMP HOUSE, WIND TURBINE,
GAS METER HOUS, HTG PLNT, PWR PLNT, UTLY PLNT, INCINERATOR, HTG PLANT, HEAT
PLANT, STANDBY GEN, PLNT, PWR HSE, UTIL PLA.

\subsection{Retrofits}
The retrofit records are from four excel files acquired from the GSA team:
``Light-Touch\_M\&V'', ``Portfolio HPGB Dashboard'' and
``GSAlink\_Buildings\_First\_55'', and ``ScopePortfolioReport\_20160105-7''.

There are 6 types of retrofits related to energy consumption reductions:
``Advanced Metering'', ``Building Envelope'', ``Building Tuneup or Utility
Improvements'', ``HVAC'', ``Lighting''. The ``Renewable Energy'', and ``Water''
type retrofits are left out from the analysis. The ``IEQ'' (indoor environment
quality) category has the following sub-categories: ``Indoor Daylighting or
Lighting Strategies'', and ``Thermal Comfort and Ventilation Measures''. We
split the IEQ retrofits and reclassified the ones with the ``Indoor Daylighting
or Lighting Strategies'' subcategory into ``Lighting'' retrofits, and the ones
with ``Thermal Comfort and Ventilation Measures'' categories into the ``HVAC''
retrofits. The above retrofit types involves modification of the building
physical environment, and are thus considered capital / hardware retrofit. An
operational / software retrofit, GSALink is also included in the analysis.
GSALink is a building fault detection and diagnostics system (FDD)~\cite{intelligentBuildings2019}.

\subsection{Built year}
The ``Entire GSA Building Portfolio.xls'' file provided to us has a built year
column, however, as the record uses a two-digit year representation, it is
unknown which century the built year indicates, thus is not usable. Currently we
resort to the list of historic buildings~\cite{GSAhistoricBuilding2020}. There
are 402 historic buildings in the portfolio. Among these buildings, some still
don't have a record of the built year. The built years among the historic
buildings with built years ranges from 1809 to 1969, with a mean of 1921.

\subsection{LEED}
LEED award date and level are retrieved from ~\cite{GBIG2020}. The records here
are given with building name and address. We match LEED project to building ID
by their addresses. Some addresses are matched to multiple buildings, we resolve
this by manually look up building names and see if it matches the project
description.

\subsection{Building Name} This field is not directly used in the analysis, but
is used as a secondary field to verify address matches, or confirm building
types. The source of building names include several excel file from the GSA
team, including ``Covered\_Facilities\_All\_Energy\_mmBTUs'',
``FY17WeatherNorm\_CMU\_BldgsOnly'', ``Entire\_GSA\_Building\_Portfolio'',
``euas\_database\_of\_buildings\_cmu'', ``LEED\_EB\_Building\_totals\_and\_years'',
``GSAlink\_Buildings\_First\_55\_Opiton\_26\_Start\_Stop\_Dates'', and some file from
some Github repo~\cite{gsaHackathon2015}.

\subsection{Combining different fields into the retrofit analysis data set}

Following table illustrate the cleaning steps and the remaining records after
each cleaning step. For step 4, we restricted building types which have more
than 10 buildings in both retrofitted and non-retrofitted group. We observed
that after step 5, the leased buildings are only in the non-retrofitted group,
as a result, we cannot learn retrofit effect for those buildings, thus in step 6
we restricted the study to owned portfolio. 

\begin{table}[H]
  \caption{retrofit data joined with monthly consumption}
  \centering
  \fontsize{10}{12}\selectfont
  \begin{tabular}{l|p{5cm}|p{3cm}|p{3cm}}
    \toprule
    step & operation & building count &building count\\
         &           & retrofitted & non-retrofitted\\
    \midrule
    0& Initial & 310 & \\
    1& With energy data & 309 & 2858 \\
    2& With energy during both pre and post retrofit period & 308 &1155 \\
    3& With more than 12 months of data for both the pre and post period & 294 & 720 \\
    4& Building type is office or courthouse & 287 & 529\\
    5& Building ownership stays the same after the retrofit & 287 & 523\\
    6& Restrict to owned portfolio & 287 & 296 \\
    7& Has climate normal & 286 & 296 \\
    8& Has monthly weather & 286 & 296 \\
    9& Has three-year pre-retrofit monthly weather & 280 & 287 \\
    10& Has climate change scenario data& 270 & 282 \\
    11 & Propensity score above 0.05, below 0.95 & 255 & 274 \\
    \bottomrule
  \end{tabular}
  \label{tab:retrofit_clean}
\end{table}
Before the removal of small propensity score data points, we examined the
average monthly energy consumption for each fuel types.

All coal consumption for the study set are zero, so we disregard the coal
consumption. The oil consumption are peaking at the right timing but the amount
of consumption is way larger than normal building consumption. We will not
consider the oil consumption in our analysis for now.

\begin{figure}[H]
  \centering
  \includegraphics[width=0.9\textwidth]{../images/retro_avg_monthly_kbtu_OIL_by_building.png}
  \caption[Oil consumption trend]{Oil consumption for various month of a year}
  \label{fig:retro_avg_monthly_kbtu_OIL_by_building}
\end{figure}

There are 15 buildings (20 projects) were awarded LEED over one year before the
retrofit completion date.

For the retrofitted buildings, we consider the two years prior to the documented
``Substantial Completion Date'' as the retrofit implementation period, and
remove them from the analysis. The three-year before the implementation period,
and the three-year after the ``Substantial Completion Date'' are used to compute
the outcome variable, the change of average monthly consumption (or consumption
per sqft). EUAS energy data has only month and year, but not a specific
duration, we will assume the center of the billing period being the 15th. This
center date will be used to filter the appropriate pre and post retrofit period
when joining retrofit time with energy data.

For the non-retrofitted buildings, a pseudo retrofit date is chosen by randomly
sampling from the actual retrofit dates in the retrofitted buildings, and the pre
and post period are computed similarly as the retrofitted buildings.

We filtered buildings with at least 12 monthly post retrofit energy data, and
three year of pre-retrofit data.

For each period, we use the majority of the building category as the category
for this period. The ties are broken randomly. After checking, this won't affect
the derived ownership (no ties appear between category C/D or other categories).
\section{Appendix II: Climate models}
We used the BCCAv2 output of the following 19 climate models

\begin{table}[H]
  \caption{Climate models}
  \centering
  \fontsize{9}{12}\selectfont
  \begin{tabular}{lp{8cm}l}
    \toprule
Modeling Center& Institution &Model\\
CSIRO-BOM &CSIRO (Commonwealth Scientific and Industrial Research Organisation,
Australia) and BOM (Bureau of Meteorology, Australia) & ACCESS1.0 \\
BCC& Beijing Climate Center, China Meteorological Administration & BCC-CSM1.1 \\
CCCma & Canadian Centre for Climate Modelling and Analysis & CanESM2 \\
NCAR & National Center for Atmospheric Research &CCSM4 \\
NSF-DOE-NCAR & National Science Foundation, Department of Energy, and National Center for Atmospheric Research &CESM1(BGC)\\
CMCC & Centro Euro-Mediterraneo per I Cambiamenti Climatici & CMCC-CM\\
CNRM-CERFACS & Centre National de Recherches Meteorologiques/Centre Europeen de
Recherche et Formation Avancees en Calcul Scientifique & CNRM-CM5\\
CSIRO-QCCCE & Commonwealth Scientific and Industrial Research Organisation in
collaboration with the Queensland Climate Change Centre of Excellence & CSIRO-Mk3.6.0 \\
NOAA GFDL & Geophysical Fluid Dynamics Laboratory & GFDL-ESM2G \\
          & Geophysical Fluid Dynamics Laboratory & GFDL-ESM2M \\
INM & Institute for Numerical Mathematics & INM-CM4 \\
IPSL & Institute Pierre-Simon Laplace & IPSL-CM5A-MR \\
&& IPSL-CM5B-LR \\
MIROC & Atmosphere and Ocean Research Institute (The University of Tokyo),
National Institute for Environmental Studies, and Japan Agency for Marine-Earth
Science and Technology & MIROC5\\
MIROC & Japan Agency for Marine-Earth Science and Technology, Atmosphere and
Ocean Research Institute (The University of Tokyo), and National Institute for
Environmental Studies & MIROC-ESM \\
&&MIROC-ESM-CHEM\\ 
MPI-M & Max Planck Institute for Meteorology (MPI-M) & MPI-ESM-LR\\ 
&&MPI-ESM-MR\\ 
MRI & Meteorological Research Institute & MRI-CGCM3\\
NCC & Norwegian Climate Centre & NorESM1-M \\
    \bottomrule
  \end{tabular}
  \label{tab:climate_models}
\end{table}
\newpage
\bibliographystyle{plain}
\bibliography{myCitation}
\end{document}